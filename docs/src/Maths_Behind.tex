\documentclass{article}

\usepackage{amsmath}
\usepackage{parskip}
\usepackage{amsfonts} % For the special mathematical operations.

\newcommand\hr{\par\vspace{-.5\ht\strutbox}\noindent\hrulefill\par}

\title{ID SYSTEM FOR PROJECT FROST}

\author{Baran Gokalp}

\date{2022 January 13 to 15} 

\begin{document}
    \maketitle
    The algorithm for the id system of project frost is just 
    some mathematical operations. First of all, we need get the 
    length of the username and assign it to the variable x.
    \hr
    \begin{center}
        1)\\
        $f:\mathbb{R} \longrightarrow \mathbb{R};$\\
        $g:\mathbb{R} \longrightarrow \mathbb{R};$\\
        $z:=len(username); n=1; i = 100000;$\\
        \hr  
        2)
        \begin{equation*}
            f(x)=\begin{cases}
                \prod_{n}^{x}n \quad &\text{if} \, x \leq 8 \\
                \sum_{n}^{x}n \quad &\text{if} \, x>8 \\
            \end{cases}
        \end{equation*}
        \hr
        3)\\
        $g(x) = f(z) + i$
    \end{center}
    \hr
    So that is what we are doing in the Id system. First we are 
    taking the username from the user and calculating it's length.
    After we calculate the length of the username, we are
    creating a piecewise function contains two conditions.
    Our critical point is 8. If length of the username is less 
    or equal to 8, then we product all the numbers 
    from 1 to length of the username. If length of the username
    is greater than 8, then we sum all the number from 1
    to length of the username. And at the end, 
    we have a function called $g(x)$. What it does is actually
    quite simple. It calls the function $f(x)$ with the length of
    the username, and it returns the sum of the f function and 
    basenumber.  
\end{document}

% End. That was it.